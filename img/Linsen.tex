\documentclass{article}
\usepackage{pst-optexp}  % Load the pst-optexp package
\usepackage{auto-pst-pdf}  % This package allows compiling PSTricks with pdflatex
\usepackage{amsmath}
\usepackage{geometry}

\geometry{paperwidth=11in, paperheight=3in}
\pagestyle{empty}
\begin{document}
\noindent
\centering
\begin{tabular}{|c|c|c|c|c|c|c|}
\hline
\textbf{Linsenform} & 
\begin{pspicture}(2,-1)
    \lens[lensradiusleft=1.5, lensradiusright=1.5, lensheight=1.5](0,0)(2,0)
\end{pspicture} & 
\begin{pspicture}(2,-1)
    \lens[lensradiusleft=0, lensradiusright=1.5, lensheight=1.5](0,0)(2,0)
\end{pspicture} & 
\begin{pspicture}(2,-1)
    \psarc(-0.3,0){1.4}{-40}{40}
    \psarc(0.4,0){1}{-70}{70}
\end{pspicture} & 
\begin{pspicture}(2,-1)
    \lens[lensradiusleft=-1.5, lensradiusright=-1.5, lensheight=1.5](0,0)(2,0)
\end{pspicture} & 
\begin{pspicture}(2,-1)
    \lens[lensradiusleft=0, lensradiusright=-1.5, lensheight=1.5](0,0)(2,0)
\end{pspicture} &
\begin{pspicture}(2,-1)
    \lens[lensradiusleft=-1.0, lensradiusright=2.5, lensheight=1.5](0,0)(2,0)
\end{pspicture}
\rule{0pt}{2.5cm}
\\
\hline
\textbf{Bezeichnung} & bikonvex & plankonvex & konkav-konvex & bikonkav & plankonkav & konvex-konkav \\
\hline
\textbf{Radien} & 
$r_1 > 0$ \, $r_2 < 0$ & 
$r_1 = \infty$ \, $r_2 < 0$ & 
$r_1 < r_2 < 0$ & 
$r_1 < 0$ \, $r_2 > 0$ & 
$r_1 = \infty$ \, $r_2 > 0$ & 
$r_2 < r_1 < 0$ \\
\hline
\textbf{Brennweite} & 
$f' > 0$ & 
$f' > 0$ & 
$f' > 0$ & 
$f' < 0$ & 
$f' < 0$ & 
$f' < 0$ \\
\hline
\end{tabular}

\end{document}
