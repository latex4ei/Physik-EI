\documentclass{standalone}
\usepackage{tikz}
\usepackage{amsmath}

\begin{document}

\begin{tikzpicture}
  % Erde zeichnen
  \draw[] (0,0) circle (2.5cm);

  % Äquator zeichnen
  \draw[dashed] (-2.5,0) -- (2.5,0);

  % Rotationsachse
  \draw[->, thick] (0,-3.5) -- (0,3.5) node[above] {$\omega$};

  % Punkt am Äquator
  \coordinate (P) at (1,0);
  \filldraw[red] (P) circle (2pt) node[anchor=west] {P};

  % Geschwindigkeitsvektor
  \draw[->, thick] (P) -- ++(1,0) node[above] {$\vec{v}$};

  % Corioliskraft Vektor
  \draw[->, thick, blue, densely dotted] (P) -- ++(0,0.5) node[right] {$\vec{F}_C$};


  % Himmelsrichtungen
  \node at (-3,0) {W};
  \node at (3,0) {O};
  \node at (0,4) {N};
  \node at (0,-4) {S};
  
\end{tikzpicture}

\end{document}
