% % % % % % % % % % % % % % % % % % % % % % % % % % % % % % % % % % % % % % % %
% LaTeX4EI Template for Cheat Sheets                                Version 1.1
%
% Authors: Markus Hofbauer
% Contact: info@latex4ei.de
% Encode: UTF-8
% % % % % % % % % % % % % % % % % % % % % % % % % % % % % % % % % % % % % % % %


% ======================================================================
% Document Settings
% ======================================================================

% possible options: color/nocolor, english/german, threecolumn
% defaults: color, english
\documentclass[german]{latex4ei/latex4ei_sheet}

% set document information
\title{Physik}
\author{LaTeX4EI}                    % optional, delete if unchanged
\myemail{info@latex4ei.de}           % optional, delete if unchanged
\mywebsite{www.latex4ei.de}          % optional, delete if unchanged


% ======================================================================
% Begin
% ======================================================================
\begin{document}

% Title
% ----------------------------------------------------------------------
\maketitle   % requires ./img/Logo.pdf


% Allgemein
% ----------------------------------------------------------------------
\section{Allgemeines}

% Klassische Mechanik
% ----------------------------------------------------------------------
\section{Klassische Mechanik}
\subsection{Bewegungen}
\subsubsection{Freier Fall}
\subsubsection{Ebene}
\subsubsection{Senkrechter Wurf}
\subsubsection{Waagrechter Wurf}
\subsubsection{Schräger Wurf}
\subsection{Newton'sche Axiome}
\subsection{Kraft, Energie, Impuls}
\subsubsection{Kraft}
\subsubsection{Arbeit}
\subsubsection{Leistung}
\subsubsection{Energie}
\subsubsection{Impuls}
\subsubsection{Raketengleichung}
\subsection{Planetenbewegungen}
\subsubsection{Gravitation}
\subsubsection{Keplersche Gesetze}
\subsection{Drehungen}
\subsubsection{Drehmoment}
\subsubsection{Trägheitsmoment}
\subsubsection{Drehimpuls}
\subsection{Corioliskraft}

% Harmonische Schwingungen
% ----------------------------------------------------------------------
\section{Harmonische Schwingungen}
\subsection{Frei ungedämpft}
\subsubsection{Federpendel}
\subsubsection{Fadenpendel}
\subsubsection{Torisionsschwingungen}
\subsection{Frei gedämpft}
\subsubsection{Fälle}
\subsection{Erzwungen gedämpft}
\subsubsection{Resonanzfälle}

% Wellen
% ----------------------------------------------------------------------
\section{Wellen}
\subsection{Allgemeines}
\subsection{Doppler-Effekt}
\subsection{Schwebung}
\subsection{Gekoppelte Wellen}
\subsection{Interferenz}
\subsubsection{Doppelspalt}
\subsection{Reflexion/Wellen in einem Medium}

% Optik
% ----------------------------------------------------------------------
\section{Optik}
\subsection{Rexlexion und Brechnung}
\subsection{Linsen}
\subsubsection{Hohlspiegel}
\subsubsection{Dünne Linsen}
\subsubsection{Dicke Linsen}
\subsubsection{Linsensysteme}
\subsection{weitere Gleichungen}
%Malus, Lambert-Beer, etc.

% Hydromechanik
% ----------------------------------------------------------------------
\section{Hydromechanik}

% Thermodynamik
% ----------------------------------------------------------------------
\section{Thermodynamik}
%aus alter LaTeX FS + Entropie und ein paar kleine Änderungen

% Quantenmechanik
% ----------------------------------------------------------------------
\section{Quantenmechanik}
\subsection{Stefan Boltzmann}
\subsection{Weiteres zur Quantenmechanik}

% ======================================================================
% End
% ======================================================================
\end{document}
